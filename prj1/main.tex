% 这是中国科学院大学计算机科学与技术专业《计算机组成原理(研讨课)》使用的实验报告 Latex 模板
% 本模板与 2024 年 2 月 Jun-xiong Ji 完成, 更改自由 Shing-Ho Lin 和 Jun-Xiong Ji 于 2022 年 9 月共同完成的基础物理实验模板
% 如有任何问题, 请联系: jijunxoing21@mails.ucas.ac.cn
% This is the LaTeX template for report of Experiment of Computer Organization and Design courses, based on its provided Word template. 
% This template is completed on Febrary 2024, based on the joint collabration of Shing-Ho Lin and Junxiong Ji in September 2022. 
% Adding numerous pictures and equations leads to unsatisfying experience in Word. Therefore LaTeX is better. 
% Feel free to contact me via: jijunxoing21@mails.ucas.ac.cn

\documentclass[11pt]{article}

\usepackage[a4paper]{geometry}
\geometry{left=2.0cm,right=2.0cm,top=2.5cm,bottom=2.5cm}

\usepackage{ctex} % 支持中文的LaTeX宏包
\usepackage{amsmath,amsfonts,graphicx,subfigure,amssymb,bm,amsthm,mathrsfs,mathtools,breqn} % 数学公式和符号的宏包集合
\usepackage{algorithm,algorithmicx} % 算法和伪代码
\usepackage[noend]{algpseudocode} % 算法和伪代码
\usepackage{fancyhdr} % 自定义页眉页脚
\usepackage[framemethod=TikZ]{mdframed} % 创建带边框的框架
\usepackage{fontspec} % 字体设置
\usepackage{adjustbox} % 调整盒子大小
\usepackage{fontsize} % 设置字体大小
\usepackage{tikz,xcolor} % 绘制图形和使用颜色
\usepackage{multicol} % 多栏排版
\usepackage{multirow} % 表格中合并单元格
\usepackage{pdfpages} % 插入PDF文件
\usepackage{listings} % 在文档中插入源代码
\usepackage{wrapfig} % 文字绕排图片
\usepackage{bigstrut,multirow,rotating} % 支持在表格中使用特殊命令
\usepackage{booktabs} % 创建美观的表格
\usepackage{circuitikz} % 绘制电路图
\usepackage{zhnumber} % 中文序号(用于标题)
\usepackage{tabularx} % 表格折行

\definecolor{dkgreen}{rgb}{0,0.6,0}
\definecolor{gray}{rgb}{0.5,0.5,0.5}
\definecolor{mauve}{rgb}{0.58,0,0.82}
\lstset{
  frame=tb,
  aboveskip=3mm,
  belowskip=3mm,
  showstringspaces=false,
  columns=flexible,
  framerule=1pt,
  rulecolor=\color{gray!35},
  backgroundcolor=\color{gray!5},
  basicstyle={\small\ttfamily},
  numbers=none,
  numberstyle=\tiny\color{gray},
  keywordstyle=\color{blue},
  commentstyle=\color{dkgreen},
  stringstyle=\color{mauve},
  breaklines=true,
  breakatwhitespace=true,
  tabsize=3,
}

% 超链接支持 (一定要在大多数包之后, cleveref 之前)
% colorlinks=true 使链接文字着色而不是加方框; hidelinks 可去掉颜色
\usepackage[colorlinks=true,linkcolor=blue,citecolor=blue,urlcolor=magenta]{hyperref}

% 轻松引用, 可以用\cref{}指令直接引用, 自动加前缀. 需要在 hyperref 之后加载
% 例: 图片label为fig:1  => \cref{fig:1} -> Figure.1; \ref{fig:1} -> 1 (纯数字)
\usepackage[capitalize]{cleveref}
% \crefname{section}{Sec.}{Secs.}
\Crefname{section}{Section}{Sections}
\Crefname{table}{Table}{Tables}
\crefname{table}{Table.}{Tabs.}

% \setmainfont{Palatino Linotype.ttf}
% \setCJKmainfont{SimHei.ttf}
% \setCJKsansfont{Songti.ttf}
% \setCJKmonofont{SimSun.ttf}
\punctstyle{kaiming}
% 偏好的几个字体, 可以根据需要自行加入字体ttf文件并调用

\renewcommand{\emph}[1]{\begin{kaishu}#1\end{kaishu}}

% 对 section 等环境的序号使用中文
\renewcommand \thesection{\zhnum{section}、}
\renewcommand \thesubsection{\arabic{section}.\arabic{subsection}}


%%%%%%%%%%%%%%%%%%%%%%%%%%%
%改这里可以修改实验报告表头的信息
\newcommand{\name}{寇逸欣}
\newcommand{\studentNum}{2023K8009922004}
\newcommand{\major}{计算机科学与技术}
\newcommand{\labNum}{01}
\newcommand{\labName}{互联网协议实验}
%%%%%%%%%%%%%%%%%%%%%%%%%%%

\begin{document}

\begin{center}
  \LARGE \bf 中国科学院大学 \\《计算机网络(研讨课)》实验报告
\end{center}

\begin{center}
  \emph{姓名} \underline{\makebox[7em][c]{\name}} 
  % 如果名字比较长, 可以修改box的长度"8em"为其他值
  \emph{学号} \underline{\makebox[12em][c]{\studentNum}}
  \emph{专业} \underline{\makebox[15em][c]{\major}}\\
  \emph{实验项目编号} \underline{\makebox[3em][c]{\labNum}}
  \emph{实验名称} \underline{\makebox[30em][c]{\labName}}\\
\end{center}

% \begin{center}
%   \begin{tabularx}{\textwidth}{|lX|}
%     \hline
%     注1: & 撰写此 Word 格式实验报告后以 PDF 格式保存 SERVE CloudIDE 的 \texttt{/home/serve-ide/ cod-lab/reports} 目录下(注意:reports 全部小写)。文件命名规则:\texttt{prjN.pdf},其中 \texttt{prj} 和后缀名 \texttt{pdf} 为小写,\texttt{N} 为1至4的阿拉伯数字。例如:\texttt{prj1.pdf}。PDF 文件大小应控制在 5MB 以内。此外,实验项目5包含多个选做内容,每个选做实验应提交各自的实验报告文件,文件命名规则:\texttt{prj5-projectname.pdf},其中``-''为英文标点符号的短横线。文件命名举例:\texttt{prj5-dma.pdf}。具体要求详见实验项目5讲义。 \\

%     注2: & 使用\texttt{git add}及\texttt{git commit}命令将实验报告\texttt{PDF}文件添加到本地仓库master分支,并通过\texttt{git push}推送到实验课SERVE GitLab远程仓库master分支(具体命令详见实验报告)。 \\

%     注3: & 实验报告模板下列条目仅供参考,可包含但不限定如下内容。实验报告中无需重复描述讲义中的实验流程。\\
%     \hline
%   \end{tabularx}
% \end{center}

  

\section{实验一:互联网协议实验}
\subsection{实验内容}
本实验主要包括以下内容:
\begin{enumerate}
  \item 在节点h1上开启wireshark抓包,用wget下载www.ucas.ac.cn页面。
  \item 调研说明wireshark抓到的几种协议——ARP, DNS, TCP, HTTP, HTTPS。
  \item 调研解释h1下载ucas页面的整个过程几种协议的运行机制。
\end{enumerate}

\subsection{实验步骤}
实验环境:virtualbox + ubuntu24.04 server + mininet + wireshark

由于使用ubuntu server版本,默认没有图形界面,因此采用终端命令行操作。并将抓包结果保存为.cap文件,之后在本地电脑上使用wireshark打开分析。在终端中输入以下命令进行实验:
\begin{lstlisting}
sudo apt install wireshark xterm ifupdown
sudo mn --nat
mininet> h1 echo "nameserver 1.2.4.8" > /etc/resolv.conf
mininet> h1 wget www.ucas.ac.cn
\end{lstlisting}

\subsection{实验结果}
\begin{figure}[H]
  \centering
  \includegraphics[width=0.8\textwidth]{fig/wireshark.png}
  \caption{Wireshark抓包结果}
  \label{fig:wireshark}
\end{figure}

\begin{figure}[H]
  \centering
  \includegraphics[width=0.8\textwidth]{fig/html.png}
  \caption{html文件}
  \label{fig:html}
\end{figure}

如图\ref{fig:wireshark}\,所示,抓包结果中包含了ARP, DNS, TCP, HTTP, HTTPS等多种协议。

\subsection{实验分析}
\subsubsection{ARP协议}
ARP协议(Address Resolution Protocol,地址解析协议)用于将网络层地址(如IP地址)解析为数据链路层地址(如MAC地址)。在局域网中,当一个主机需要与另一个主机通信时,
它需要知道对方的MAC地址。 ARP协议通过广播ARP请求来实现地址解析。请求包含发送方的IP地址和MAC地址,以及目标主机的IP地址。所有接收到ARP请求的主机都会检查请求中的目标IP地址,
如果匹配自己的IP地址,则会发送一个ARP响应,包含自己的MAC地址。发送方接收到ARP响应后,就可以将目标主机的MAC地址缓存起来,以便后续通信使用。

\begin{figure}[H]
  \centering
  \includegraphics[width=0.8\textwidth]{fig/arp.png}
  \caption{ARP协议}
  \label{fig:arp}
\end{figure}

\ref{fig:arp}\, 图中展示了ARP协议的工作流程。主机A需要与主机B通信,但不知道主机B的MAC地址。于是主机A广播一个ARP请求,询问谁拥有IP地址10.0.0.3。
主机B收到请求后,发现请求中的目标IP地址与自己的IP地址匹配,于是发送一个ARP响应,告诉主机A自己的MAC地址为4a:c9:ff:61:78:25。主机A收到响应后,就可以使用该MAC地址与主机B进行通信。

\subsubsection{DNS协议}
DNS协议(Domain Name System,域名系统)用于将域名解析为IP地址。在互联网中,主机之间的通信是通过IP地址进行的,但为了方便用户记忆,通常使用域名来代替IP地址。
当用户在浏览器中输入一个域名时,计算机会首先向DNS服务器发送一个DNS请求,请求解析该域名。DNS服务器收到请求后,会查找该域名对应的IP地址,并将结果返回给客户端。
客户端收到响应后,就可以使用该IP地址与目标主机进行通信。

\begin{figure}[H]
  \centering
  \includegraphics[width=0.8\textwidth]{fig/dns.png}
  \caption{DNS协议}
  \label{fig:dns}
\end{figure}

我们可以选择 Follow UDP Stream 来查看 DNS 请求和响应的详细内容,如图 \ref{fig:dns_stream} 所示。
10.0.0.1 向 1.2.4.8 发送 DNS 查询,请求解析 www.ucas.ac.cn 的 IPv4(A 记录)和 IPv6(AAAA 记录)地址。1.2.4.8 向 10.0.0.1 返回 DNS 响应,
响应中包含了 www.ucas.ac.cn 的 A 记录(IPv4 地址)、NS 记录(域名服务器)、SOA 记录(起始授权机构)等信息。
查询和响应中均显示了 DNS 查询的 ID、类型(A/AAAA/SOA/NS)、目标域名和解析结果。

\begin{figure}[H]
  \centering
  \includegraphics[width=0.8\textwidth]{fig/dns_stream.png}
  \caption{DNS请求和响应的详细内容}
  \label{fig:dns_stream}
\end{figure}

\subsubsection{TCP协议}
TCP协议(Transmission Control Protocol,传输控制协议)是一种面向连接的、可靠的传输层协议。它为应用层提供了可靠的数据传输服务,确保数据包按照顺序到达目的地,
并且在丢失或损坏的情况下能够进行重传。TCP协议通过三次握手建立连接,通过四次挥手关闭连接。在数据传输过程中,TCP协议会对数据进行分段,并为每个数据段分配序列号,
以便接收方能够按照正确的顺序重组数据。

\begin{figure}[H]
  \centering
  \includegraphics[width=0.8\textwidth]{fig/tcp.png}
  \caption{TCP协议}
  \label{fig:tcp}
\end{figure}

图 \ref{fig:tcp_stream} 展示了一个完整的 HTTP 请求与响应,以及 TCP 连接的建立和正常关闭过程,具体如下:
\begin{itemize}
  \item \textbf{TCP三次握手建立连接}
  \begin{enumerate}
    \item 10.0.0.1 → 124.16.77.5:SYN
    \item 124.16.77.5 → 10.0.0.1:SYN, ACK
    \item 10.0.0.1 → 124.16.77.5:ACK
  \end{enumerate}

  \item \textbf{HTTP请求与响应}
  \begin{enumerate}
    \item 10.0.0.1 → 124.16.77.5:发送 HTTP GET 请求
    \item 124.16.77.5 → 10.0.0.1:返回 HTTP/1.1 301 Moved Permanently(网页永久重定向)
  \end{enumerate}

  \item \textbf{TCP四次挥手正常断开连接}
  \begin{enumerate}
    \item 10.0.0.1 → 124.16.77.5:ACK
    \item 10.0.0.1 → 124.16.77.5:FIN, ACK(主动关闭连接)
    \item 124.16.77.5 → 10.0.0.1:ACK
    \item 124.16.77.5 → 10.0.0.1:FIN, ACK(被动关闭连接)
    \item 10.0.0.1 → 124.16.77.5:ACK
  \end{enumerate}
\end{itemize}

\begin{figure}[H]
  \centering
  \includegraphics[width=0.8\textwidth]{fig/tcp_stream.png}
  \caption{TCP连接的详细内容}
  \label{fig:tcp_stream}
\end{figure}

\subsubsection{HTTP协议}
HTTP协议(Hypertext Transfer Protocol,超文本传输协议)是应用层协议,用于在客户端和服务器之间传输超文本数据。当用户在浏览器中输入一个URL时,
浏览器会向服务器发送一个HTTP请求,请求获取指定的资源。服务器收到请求后,会返回一个HTTP响应,包含请求的资源和状态码。HTTP协议是无状态的,即每个请求都是独立的,
服务器不会记录客户端的状态信息。

\begin{figure}[H]
  \centering
  \includegraphics[width=0.8\textwidth]{fig/http.png}
  \caption{HTTP协议}
  \label{fig:http}
\end{figure}

\subsubsection{HTTPS协议}
HTTPS协议(Hypertext Transfer Protocol Secure,安全超文本传输协议)是在HTTP协议的基础上,加入了SSL/TLS协议进行加密和身份验证。
它通过在客户端和服务器之间建立一个安全的加密通道,确保数据在传输过程中的安全性和完整性。HTTPS协议广泛应用于在线支付、网上银行等对安全性要求较高的场景。

\section{实验二:流完成时间实验}

\subsection{实验内容}
本实验主要包括以下内容:
\begin{enumerate}
  \item 利用fct_exp.py脚本复现上页幻灯片中的图,每个数据点做3次实验,取均值
  \item 调研解释图中的现象,提示:TCP传输、慢启动机制
\end{enumerate}

\subsection{实验数据}
\begin{table}[H]
  \centering
  \caption{流完成时间实验数据}
  \label{tab:fct_data}
  \begin{tabular}{cccccc}
    \toprule
    \textbf{Size (MB)} & \textbf{10.0 Mbps} & \textbf{50.0 Mbps} & \textbf{100.0 Mbps} & \textbf{500.0 Mbps} & \textbf{1000.0 Mbps} \\
    \midrule
    1.0  & 1.154964 & 0.721915 & 0.798944 & 0.792040 & 0.793336 \\
    10.0 & 8.696521 & 2.234174 & 1.484441 & 1.046817 & 1.100703 \\
    100.0 & 84.854350 & 17.423328 & 9.137862 & 2.552476 & 1.797471 \\
    \bottomrule
  \end{tabular}
\end{table}

根据表\ref{tab:fct_data}\,中的数据,我们可以绘制出流完成时间与带宽的关系图,如图\ref{fig:fct_plot}所示。
\begin{figure}[H]
  \centering
  \includegraphics[width=0.8\textwidth]{fig/fct_plot.png}
  \caption{流完成时间与带宽的关系图}
  \label{fig:fct_plot}
\end{figure}

\subsection{实验分析}
从图\ref{fig:fct_plot}\,中可以看出,随着带宽的增加,流完成时间呈现出下降的趋势。这是因为更高的带宽允许更多的数据在单位时间内传输,从而减少了传输所需的时间。
然而,流完成时间的下降并不是线性的,尤其是在较低带宽下,流完成时间的变化较为显著,而在较高带宽下,流完成时间的变化趋于平缓。
这种现象可以通过TCP的慢启动机制来解释。TCP在连接建立初期采用慢启动算法,初始拥塞窗口较小,随着数据的成功传输,拥塞窗口逐渐增大,从而提高了传输速率。
在低带宽下,慢启动阶段占据了较大的比例,因此流完成时间变化较大。而在高带宽下,慢启动阶段相对较短,传输速率更快,流完成时间的变化趋于平缓。
此外,随着带宽的增加,其他因素如延迟和协议开销可能会对流完成时间产生更大的影响,从而导致流完成时间的下降幅度减小。

\end{document}