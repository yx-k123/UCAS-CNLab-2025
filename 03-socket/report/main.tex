% 这是中国科学院大学计算机科学与技术专业《计算机组成原理(研讨课)》使用的实验报告 Latex 模板
% 本模板与 2024 年 2 月 Jun-xiong Ji 完成, 更改自由 Shing-Ho Lin 和 Jun-Xiong Ji 于 2022 年 9 月共同完成的基础物理实验模板
% 如有任何问题, 请联系: jijunxoing21@mails.ucas.ac.cn
% This is the LaTeX template for report of Experiment of Computer Organization and Design courses, based on its provided Word template. 
% This template is completed on Febrary 2024, based on the joint collabration of Shing-Ho Lin and Junxiong Ji in September 2022. 
% Adding numerous pictures and equations leads to unsatisfying experience in Word. Therefore LaTeX is better. 
% Feel free to contact me via: jijunxoing21@mails.ucas.ac.cn

\documentclass[11pt]{article}

\usepackage[a4paper]{geometry}
\geometry{left=2.0cm,right=2.0cm,top=2.5cm,bottom=2.5cm}

\usepackage{ctex} % 支持中文的LaTeX宏包
\usepackage{amsmath,amsfonts,graphicx,subfigure,amssymb,bm,amsthm,mathrsfs,mathtools,breqn} % 数学公式和符号的宏包集合
\usepackage{algorithm,algorithmicx} % 算法和伪代码
\usepackage[noend]{algpseudocode} % 算法和伪代码
\usepackage{fancyhdr} % 自定义页眉页脚
\usepackage[framemethod=TikZ]{mdframed} % 创建带边框的框架
\usepackage{fontspec} % 字体设置
\usepackage{adjustbox} % 调整盒子大小
\usepackage{fontsize} % 设置字体大小
\usepackage{tikz,xcolor} % 绘制图形和使用颜色
\usepackage{multicol} % 多栏排版
\usepackage{multirow} % 表格中合并单元格
\usepackage{pdfpages} % 插入PDF文件
\usepackage{listings} % 在文档中插入源代码
\usepackage{wrapfig} % 文字绕排图片
\usepackage{bigstrut,multirow,rotating} % 支持在表格中使用特殊命令
\usepackage{booktabs} % 创建美观的表格
\usepackage{circuitikz} % 绘制电路图
\usepackage{zhnumber} % 中文序号(用于标题)
\usepackage{tabularx} % 表格折行

\definecolor{dkgreen}{rgb}{0,0.6,0}
\definecolor{gray}{rgb}{0.5,0.5,0.5}
\definecolor{mauve}{rgb}{0.58,0,0.82}
\lstset{
  frame=tb,
  aboveskip=3mm,
  belowskip=3mm,
  showstringspaces=false,
  columns=flexible,
  framerule=1pt,
  rulecolor=\color{gray!35},
  backgroundcolor=\color{gray!5},
  basicstyle={\small\ttfamily},
  numbers=none,
  numberstyle=\tiny\color{gray},
  keywordstyle=\color{blue},
  commentstyle=\color{dkgreen},
  stringstyle=\color{mauve},
  breaklines=true,
  breakatwhitespace=true,
  tabsize=3,
}

% 超链接支持 (一定要在大多数包之后, cleveref 之前)
% colorlinks=true 使链接文字着色而不是加方框; hidelinks 可去掉颜色
\usepackage[colorlinks=true,linkcolor=blue,citecolor=blue,urlcolor=magenta]{hyperref}

% 轻松引用, 可以用\cref{}指令直接引用, 自动加前缀. 需要在 hyperref 之后加载
% 例: 图片label为fig:1  => \cref{fig:1} -> Figure.1; \ref{fig:1} -> 1 (纯数字)
\usepackage[capitalize]{cleveref}
% \crefname{section}{Sec.}{Secs.}
\Crefname{section}{Section}{Sections}
\Crefname{table}{Table}{Tables}
\crefname{table}{Table.}{Tabs.}

% \setmainfont{Palatino Linotype.ttf}
% \setCJKmainfont{SimHei.ttf}
% \setCJKsansfont{Songti.ttf}
% \setCJKmonofont{SimSun.ttf}
\punctstyle{kaiming}
% 偏好的几个字体, 可以根据需要自行加入字体ttf文件并调用

\renewcommand{\emph}[1]{\begin{kaishu}#1\end{kaishu}}

% 对 section 等环境的序号使用中文
\renewcommand \thesection{\zhnum{section}、}
\renewcommand \thesubsection{\arabic{section}.\arabic{subsection}}


%%%%%%%%%%%%%%%%%%%%%%%%%%%
%改这里可以修改实验报告表头的信息
\newcommand{\name}{寇逸欣}
\newcommand{\studentNum}{2023K8009922004}
\newcommand{\major}{计算机科学与技术}
\newcommand{\labNum}{02}
\newcommand{\labName}{socket应用编程实验}
%%%%%%%%%%%%%%%%%%%%%%%%%%%

\begin{document}

\input{tex_file/head.tex}

\section{实验内容}
本实验要求使用C语言实现一个简单的HTTP服务器,能够处理基本的GET请求,解析请求报文、返回相应应答及内容。具体要求如下:

\begin{table}[H]
\centering
\caption{需要支持的状态码及场景}
\begin{tabular}{|c|p{10cm}|}
\hline
\textbf{状态码} & \textbf{场景} \\ \hline
200 OK & 对于443端口接收的请求,如果程序所在文件夹存在所请求的文件,返回该状态码,以及所请求的文件 \\ \hline
301 Moved Permanently & 对于80端口接收的请求,返回该状态码,在应答中使用Location字段表达相应的https URL \\ \hline
206 Partial Content & 对于443端口接收的请求,如果所请求的为部分内容(请求中有Range字段),返回该状态码,以及相应的部分内容 \\ \hline
404 Not Found & 对于443端口接收的请求,如果程序所在文件夹没有所请求的文件,返回该状态码 \\ \hline
\end{tabular}
\end{table}

\section{实验流程}
\subsection{主函数设计}
任务要求我们同时监听80和443端口,故而想到在主函数中创建两个线程,分别调用监听端口函数,传递参数80和443。主函数代码如下:
\begin{lstlisting}[language=C]
int main()
{
    pthread_t http_thread;
    pthread_t https_thread;

    int http_port = HTTP_PORT;
    int https_port = HTTPS_PORT;

    if (pthread_create(&http_thread, NULL, listen_port, &http_port)) {
        perror("creat http thread error!\n");
        exit(1);
    }
    if (pthread_create(&https_thread, NULL, listen_port, &https_port)) {
        perror("creat https thread error!\n");
        exit(1);
    }

    pthread_join(http_thread, NULL);
    pthread_join(https_thread, NULL);

    return 0;
}
\end{lstlisting}
pthread\_create函数创建线程,传递的参数是监听端口函数listen\_port和端口号。pthread\_join函数等待线程结束。

\subsection{端口监听函数}
\begin{figure}[H]
\centering
\includegraphics[width=0.8\textwidth]{fig/flowchart.png}
\caption{客户端应答流程图}
\end{figure}

在给的代码框架中,已经有了这部分的代码,我们只需要修改一些细节。具体来说就是要设置一个端口号,
用来区分是80端口还是443端口。然后在接收到请求后,调用处理请求函数handle\_https\_request或handle\_http\_request。

\subsection{request 解析}
我们在接受到请求后,自然的想要将请求报文解析出来。故首先,我们需要设计数据结构Request来存储这些数据,包含请求方法、URL、版本号、Range字段等。
\begin{lstlisting}[language=C]
typedef struct Header {
    char *name;
    char *value;
    struct Header *next;
} Header;

typedef struct line {
    char method[8];
    char url[256];
    char version[16];
} Line;

typedef struct Request{
    Line line;
    Header *headers;
    char *body;
} Request;
\end{lstlisting}

然后我们需要设计一个函数来解析请求报文。请求报文的样式如下:
\begin{figure}[H]
\centering
\includegraphics[width=0.8\textwidth]{fig/request.png}
\caption{请求报文格式}
\end{figure}

由此可知,我们可以先按行分割请求报文,然后解析第一行,按空格分割得到请求方法、URL和版本号。
接下来解析后续的每一行,按冒号分割得到Header的name和value,存入链表中,然后是请求体。最后将解析结果存入Request结构体中。
\begin{lstlisting}[language=C]
void decode_request(char *raw_request, Request *request)
{
    char *line_end = strstr(raw_request, "\r\n");
    sscanf(raw_request, "%s %s %s", request->line.method, request->line.url, request->line.version);

    char *header_start = line_end + 2;
    char *header_end;
    Header *current_header = NULL;

    while ((header_end = strstr(header_start, "\r\n")) != NULL && header_end != header_start) {
        Header *new_header = (Header *)malloc(sizeof(Header));
        new_header->next = NULL;

        char *colon_pos = strstr(header_start, ": ");
        if (colon_pos != NULL) {
            int name_len = colon_pos - header_start;
            int value_len = header_end - (colon_pos + 2);

            new_header->name = (char *)malloc(name_len + 1);
            new_header->value = (char *)malloc(value_len + 1);

            strncpy(new_header->name, header_start, name_len);
            new_header->name[name_len] = '\0';
            strncpy(new_header->value, colon_pos + 2, value_len);
            new_header->value[value_len] = '\0';

            if (current_header == NULL) {
                request->headers = new_header;
            } else {
                current_header->next = new_header;
            }
            current_header = new_header;
        }

        header_start = header_end + 2;
    }

    if (strcmp(request->line.method, "POST") == 0) {
        request->body = strdup(header_start);
    } else {
        request->body = NULL;
    }
}
\end{lstlisting}

最终,h1所收到并解析出的请求报文如下:
\begin{figure}[H]
\centering
\includegraphics[width=0.8\textwidth]{fig/wireshark_request.png}
\caption{h1收到的请求报文}
\end{figure}

\subsection{http请求处理}
对于http请求,我们永远将其重定向到https。故而我们只需要返回301状态码,并在Location字段中拼接"https://10.0.0.1"和请求的URL即可。

最终http发送的应答报文如下:
\begin{figure}[H]
\centering
\includegraphics[width=0.8\textwidth]{fig/wireshark_http_response.png}
\caption{h1发送的http应答报文}
\end{figure}

\subsection{https请求处理}
对于https请求,我们需要根据请求的URL来判断请求的文件是否存在,若存在则返回200状态码和文件内容,若不存在则返回404状态码。
另外,我们还需要处理Range字段,若请求中包含Range字段,则返回206状态码和部分内容。

根据test文件可知,请求文件的url的格式为"/path/filename",故我们只需要在前面加上"."即可得到文件路径"./path/filename"。

对于Range字段,我们需要解析出起始位置和结束位置。不过结束位置可能不存在,若不存在则表示请求从起始位置到文件末尾的内容。

此时我们设置一个file\_pointer指向文件的起始位置,然后使用fseek函数将文件指针移动到起始位置,接着使用fread函数读取指定长度的内容。
最后将读取的内容存入响应报文中返回给客户端。

https返回的应答报文如下:
\begin{figure}[H]
\centering
\includegraphics[width=0.8\textwidth]{fig/wireshark_https_response.png}
\caption{h1发送的https应答报文}
\end{figure}

206返回的结果如下图:
\begin{figure}[H]
\centering
\includegraphics[width=0.8\textwidth]{fig/wireshark_206_response.png}
\caption{h1发送的206应答报文}
\end{figure}

由于https和http请求处理函数大部分代码相似,我们这里只展示https请求处理函数handle\_https\_request的代码。
\begin{lstlisting}[language=C]
void handle_https_request(SSL* ssl)
{   
    char *request = calloc(1024, sizeof(char));
    char *response = calloc(1024, sizeof(char));
    int request_len = 0;
    int response_len = 0;

    if (SSL_accept(ssl) == -1){
		perror("SSL_accept failed");
		exit(1);
	}

    request_len = SSL_read(ssl, request, 1024);
    if (request_len < 0) {
        perror("SSL_read failed");
        exit(1);
    }

    Request *http_request = calloc(1, sizeof(Request));
    decode_request(request, http_request);

    int option = 0; // 0: 200 OK, 1: 206 Partial Content
    FILE *file_pointer = NULL;

    for (Header *header = http_request->headers; header != NULL; header = header->next) {
        if (strcmp(header->name, "Range") == 0) {
            option = 1;
            break;
        }
    }

    // search file
    char file_path[256] = ".";
    char *file_path_pointer = file_path + 1;
    strcat(file_path, http_request->line.url);

    if ((file_pointer = fopen(file_path_pointer - 1, "r")) == NULL) {
        response_len = sprintf(response, "%s %d Not Found\r\n\r\n", http_request->line.version, NOT_FOUND);
        SSL_write(ssl, response, response_len);
    } else {
        if (option == 0) {
            // 200 OK
            fseek(file_pointer, 0, SEEK_END);
            int file_size = ftell(file_pointer);
            fseek(file_pointer, 0, SEEK_SET);

            response_len = sprintf(response, "%s %d OK\r\nContent-Length: %d\r\n\r\n", http_request->line.version, OK, file_size);
            SSL_write(ssl, response, response_len);

            char *file_buffer = (char *)malloc(file_size);
            fread(file_buffer, 1, file_size, file_pointer);
            SSL_write(ssl, file_buffer, file_size);
            free(file_buffer);
        } else if (option == 1) {
            // 206 Partial Content
            char range_value[64];
            for (Header *header = http_request->headers; header != NULL; header = header->next) {
                if (strcmp(header->name, "Range") == 0) {
                    strcpy(range_value, header->value);
                    break;
                }
            }

            int start, end;
            if (sscanf(range_value, "bytes=%d-%d", &start, &end) != 2) {
                sscanf(range_value, "bytes=%d-", &start);
                end = -1;
            }

            fseek(file_pointer, 0, SEEK_END);
            int file_size = ftell(file_pointer);
            if (end == -1 || end >= file_size) {
                end = file_size - 1;
            }
            int content_length = end - start + 1;
            fseek(file_pointer, start, SEEK_SET);

            response_len = sprintf(response, "%s %d Partial Content\r\nContent-Length: %d\r\nContent-Range: bytes %d-%d/%d\r\n\r\n", http_request->line.version, Partial_Content, content_length, start, end, file_size);
            SSL_write(ssl, response, response_len);

            char *file_buffer = (char *)malloc(content_length);
            fread(file_buffer, 1, content_length, file_pointer);
            SSL_write(ssl, file_buffer, content_length);
            free(file_buffer);
        }
        fclose(file_pointer);
    }

    return;
}
\end{lstlisting}

\section{遇到的问题}

\begin{enumerate}
  \item 在实现https请求处理函数时,最开始没有考虑到Range结束位置可能不存在的情况,导致程序在处理类似"Range: bytes=200-"的请求时出错。后来通过检查sscanf的返回值来解决这个问题。
  \item 在实现http请求处理函数时,最开始没有正确设置Location字段,导致无法正确重定向到https。后来通过在Location字段中拼接"https://10.0.0.1"来解决这个问题。
\end{enumerate}

\end{document}